\section{PS6: Random Writer}\label{sec:ps6}
\graphicspath{{ps6}}
\subsection{Discussion:}\label{sec:ps6:disc}
 The Markov Model predicts the next characters in a sequence of k length words called kgrams.
 The RandWriter class takes a string and order k as input and maps each kgram in the string to each
 character following that kgram and it's frequency.
 This class can now generate strings of any given length, using the kgrams that have been seen before.
 The string's characters are based on the probabilities of the k-length strings that were stored in the hashmap.
 \newline
 \textbf{I used the given structure in this assignment:}
 \begin{lstlisting}
 class RandWriter {
public:

RandWriter(string text, int k);
int orderK() const; // Order k of Markov model
// Number of occurences of kgram in text
// Throw an exception if kgram is not length k
int freq(string kgram) const;
// Number of times that character c follows kgram
// if order=0, return num of times that char c appears
// (throw an exception if kgram is not of length k)
int freq(string kgram, char c) const;
// Random character following given kgram
// (throw an exception if kgram is not of length k)
// (throw an exception if no such kgram)
char kRand(string kgram);

string generate(string kgram, int L);
}

 \end{lstlisting}

\subsection{Key algorithms, Data structures and OO Designs used in this Assignment:}
RandWriter's constructor takes a string and and the order of the kgrams as input.
It initializes and constructs the hashmap.
The kRand function generates a string of length n, using the hashmap that was generated.
The function just iterates through a kgram's character map (the nested, inner map) and appends them in a string.
Then it randomly returns a char from this string.
The freq function returns the frequency of a kgram by searching/traversing the string.
The second overloaded function returns the frequency of a character, which is simple and just looks it up in the hashmap;
> return ktable.at(kgram).at(c);
The generate function generates a new string by just calling the rand function, until the desired output length is reached.


\subsection{What I accomplished :}

I have accomplished to implement Markov Model. I have created a working Random Writer.

\subsection{What I learned :}

I have learnt about the Markov model and how it works. Also learnt about the Hash Map.

\subsection{Challenges :}

Implementing the Markov Model was Challenging for me.

\subsection{Codebase}\label{sec:ps6:code}

\textbf{\colorbox{pink}{Makefile :}} \newline \textbf{This Makefile contains the lint.}
\lstinputlisting[language=Make]{ps6/Makefile}

\textbf{\colorbox{pink}{TextWriter.cpp :}} \newline \textbf{This file is important as it takes command line arguments and also gives you the output.}
\lstinputlisting{ps6/TextWriter.cpp}

\textbf{\colorbox{pink}{RandWriter.h :}} \newline \textbf{The initialization of variables and methods.}
\lstinputlisting{ps6/RandWriter.h}
\newpage

\textbf{\colorbox{pink}{RandWriter.cpp :}} \newline \textbf{The important functions for the Markov Model are implemented in this file.}
\lstinputlisting{ps6/RandWriter.cpp}




\textbf{\colorbox{pink}{test.cpp :}} \newline \textbf{The test file tests for the given input as well as runtimeerror exception}
\lstinputlisting{ps6/test.cpp}


\newpage
